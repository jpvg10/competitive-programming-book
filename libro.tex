\documentclass{article}
\usepackage[spanish]{babel}
\usepackage[utf8]{inputenc}
\usepackage{listings}
\usepackage{xcolor}
\usepackage{multicol}
\usepackage{geometry}
\usepackage{amsmath}

\lstset{
	language=java,
	numberstyle=\tiny, 
	breaklines=true,
	numbersep=1pt,
	tabsize=2,
	xleftmargin=.01in,
	xrightmargin=.01in,
	postbreak=\raisebox{0ex}[0ex][0ex]{\ensuremath{\color{red}\hookrightarrow\space}},
	numbers=left
}

\geometry{landscape,letterpaper,tmargin=1cm,bmargin=2cm,lmargin=1cm,rmargin=1cm}
\setlength{\columnsep}{0.25in}
\setlength{\columnseprule}{1px}

\def\multiset#1#2{\ensuremath{\left(\kern-.3em\left(\genfrac{}{}{0pt}{}{#1}{#2}\right)\kern-.3em\right)}}

\begin{document}

\begin{multicols}{2}

\tableofcontents

\section{Mapas}
Estructura de datos que guarda pares \emph{(clave, valor)}. El HashMap no pone las claves en ningún orden en particular. TreeMap ordena las claves de acuerdo a su orden natural. LinkedHashMap pone las claves en el orden en que se ingresen.

Las operaciones .put(), .get() y .containsKey() son \( O(1) \) en HashMap y LinkedHashMap, y \( O(\log n) \) en TreeMap.

Ejemplo: Contar cuántas veces aparece cada palabra en un String.
\lstinputlisting[firstline=6, lastline=29]{./src/Mapas.java}

\section{Sets}
Estructura de datos que actúa como ``bolsa'' donde se almacenan elementos, pero no puede almacenar elementos duplicados.

En HashSet .add() y .contains() son \( O(1) \), mientras que en TreeSet son \( O(\log n) \). Sin embargo, en el TreeSet los elementos quedan ordenados.
\lstinputlisting[firstline=5, lastline=19]{./src/Sets.java}

\section{Union-Find}
Estructura de datos que soporta las siguientes operaciones eficientemente:
\begin{itemize}
	\item Unir los conjuntos de los elementos \( p, q \)
	\item Determinar si los elementos \( p, q \) pertenecen al mismo conjunto o no
\end{itemize}
\lstinputlisting[firstline=2, lastline=65]{./src/UnionFind.java}

\section{Grafos}
	\subsection{BFS y DFS}
	Recorren un grafo a partir de un nodo origen y visitan todos los nodos alcanzables desde éste. Ambos algoritmos tienen una complejidad de \( O(n + m) \) donde \( n \) es el número de nodos y \( m \) es el número de aristas del grafo. 
	
	El siguiente ejemplo está con DFS pero funciona igual con BFS.	
	\lstinputlisting[firstline=8, lastline=75]{./src/Grafos.java}

	\subsection{Shortest Hop}
	Modificación de BFS que calcula el camino más corto desde un nodo origen \( s \) a todos los demás. Sólo funciona cuando el peso de todas las aristas es 1. Su complejidad es la misma de BFS: \( O(n + m) \).
	\lstinputlisting[firstline=8, lastline=52]{./src/ShortestHop.java}
	
	\subsection{Ordenamiento topológico}
	Todo grafo dirigido acíclico (DAG) tiene un ordenamiento topológico. Esto significa que para todas las aristas \( (u,v) \), \(u\) aparece en el ordenamiento antes que \(v\). Visualmente es como si se pusieran todos los nodos en línea recta y todas las aristas fueran de izquierda a derecha, ninguna de derecha a izquierda. 
	
	El algoritmo para hallar dicho ordenamiento es una modificación de DFS y complejidad es la misma: \( O(n + m) \). El método retorna falso si detecta un ciclo en el grafo, ya que en este caso no existe ordenamiento topológico posible.
	\lstinputlisting[firstline=7, lastline=55]{./src/TopoSort.java}
	
	\subsection{Componentes fuertemente conexas (Algoritmo de Tarjan)}
	Calcula la componente fuertemente conexa a la que pertenece cada nodo de un grafo dirigido. Si dos nodos \( u, v \) están en la misma componente, significa que existe un camino de \( u \) a \( v \) y uno de \( v \) a \( u \). Su complejidad es \( O(n + m) \).
	\lstinputlisting[firstline=6, lastline=64]{./src/Tarjan.java}
	
	\subsection{Puntos de articulación}
	Halla los puntos de articulación de un grafo. Un punto de articulación es un nodo del grafo que si se quitara causaría que el grafo se ``desconectara''. Si el grafo no era conexo en un principio, un punto de articulación es un nodo que, si se quitara, incrementaría el número de componentes conexas. 
	
	La complejidad del algoritmo es \( O(n + m) \).
	\lstinputlisting[firstline=6, lastline=58]{./src/ArticulationPoints.java}
	
	\subsection{Puentes}
	Halla los puentes de un grafo.  Un puente es una arista del grafo que si se quitara causaría que el grafo se ``desconectara''. Si el grafo no era conexo en un principio, un puente es una arista que, si se quitara, incrementaría el número de componentes conexas.
	
	La complejidad del algoritmo es \( O(n + m) \).
	\lstinputlisting[firstline=3, lastline=64]{./src/GraphBridges.java}
	
	\subsection{Minimum Spanning Tree (Algoritmo de Kruskal)}
	Halla el árbol de cubrimiento mínimo de un grafo no-dirigido y conexo. Si no está garantizado de antemano que el grafo sea conexo, hay que verificarlo antes de correr este algoritmo. 
		
	Utiliza la estructura de datos Union-Find discutida anteriormente. El grafo debe ser representado como una lista de objetos tipo Arista. Tiene una complejidad de \( O( m \log n) \).	
	\lstinputlisting[firstline=4, lastline=50]{./src/Kruskal.java}
	
	\subsection{Algoritmo de Dijkstra}
	Halla la distancia más corta desde un nodo origen \emph{src} hacia todos los demás nodos. Funciona con grafos dirigidos y no dirigidos, siempre y cuando los pesos de las aristas sean no-negativos. Se debe representar el grafo tanto en lista como en matriz de adyacencia. Su complejidad es \( O(m + n \log n) \).
	\lstinputlisting[firstline=6, lastline=85]{./src/Dijkstra.java}
	
	\subsection{Algoritmo de Floyd-Warshall}
	Halla la distancia más corta desde todos los nodos hacia todos los demás. El grafo debe estar representado en matriz de adyacencia, y puede tener aristas con peso negativo. Sin embargo, no puede tener ciclos de peso negativo. En caso de que exista un ciclo negativo, el algoritmo lo detectará. Si todas las artistas son no-negativas, se puede omitir la comprobación del ciclo negativo.
	
	La matriz de adyacencia debe armarse así:
	\[
		\text{grafo}(i, j) = \left \{ 
			\begin{array}{lcc}
				0 & si & i = j
				\\ c_{i,j} & si & \text{existe una arista de } i \text{ a } j \text{ con costo } c_{i,j}
				\\ \infty & si & i \neq j \text{ y no existe arista de } i \text{ a } j 
			\end{array}
		\right.
	\]
	La complejidad del alrgoritmo es \( O(n^3) \).
	\lstinputlisting[firstline=5, lastline=55]{./src/FloydWarshall.java}

	\subsection{Máximo flujo y mínimo corte (Algoritmo de Edmonds-Karp)}
	Halla el máximo flujo que se puede emitir desde un origen \( s \) hacia un destino \( t \), dado que los enlaces (aristas) tienen una capacidad dada.
	
	El algoritmo de Edmonds-Karp es una implementación del método de Ford-Fulkerson que usa BFS para hallar los caminos en el grafo residual. Su complejidad es de \( O(nm^2) \).
	
	Variantes de este problema:	
	
	\begin{itemize}
		\item Mínimo corte: El máximo flujo es igual al mínimo corte (esto es un teorema). Por ende, este algoritmo halla también el mínimo costo de cortar aristas de manera que \( s \) y \( t \) queden desconectados.
		\item Si hay varios orígenes \( \{s_1, s_2, ...\} \) , se pone un ``super origen'' \( s \), y se agregan aristas \( (s, s_i) \) con capacidad infinita. Análogamente, si hay varios destinos \( \{t_1, t_2, ...\} \), se agrega un ``super destino'' \( t \) y se agregan aristas \( (t_i, t) \) con capacidad infinita.
		\item Nodos con capacidad: Si el nodo \( u \) tiene también una capacidad \( c_u \), se divide en dos nodos. Un nodo \( u_l\) que recibe todas las aristas que entran a \( u \), y un nodo \( u_r \) del que salen todas las aristas que salen de \( u \). Posteriormente, se agrega una arista \( (u_l, u_r) \) con capacidad \( c_u \).
	\end{itemize}
	
	\lstinputlisting[firstline=9, lastline=84]{./src/MaxFlow.java}
	
	\subsubsection{Máximo matching de un grafo bipartito}
	Un matching de un grafo es un subconjunto de aristas tal que en cada nodo incida máximo una de ellas. Si se trata de un grafo bipartito, encontrar el máximo matching (aquel con mayor cardinalidad) se puede modelar como un problema de máximo flujo: 
	
	Sea el grafo original \( G = (V, E) \), donde \( V = L \cup R \) (es decir, los vértices se separan en dos subconjuntos, ya que el grafo es bipartito). Se construye un nuevo grafo \( G' \), con los mismos vértices y aristas del grafo original. Se agregan a \( G' \) dos nuevos vértices \( s \) y \( t \). Posteriormente se agregan aristas de \( (s, l_i) \) para los vértices \( l_i  \in L \), y aristas \( (u_i, t) \) para los vértices \( u_i \in U \). Todas las aristas se ponen con capacidad 1.
	
	El máximo matching en \( G \) es equivalente al máximo flujo entre \( s \) y \( t \) en \( G' \).
	
	Esta no es la forma más eficiente de resolver este problema, ya que hay algoritmos específicos para él que no lo modelan como un máximo flujo (por ejemplo, el algoritmo de Hopcroft-Karp).

\section{KMP}
Algoritmo para buscar una cadena \emph{pattern} dentro de una cadena \emph{text}. Su complejidad es de \( O(m+n) \) donde \( m \) es la longitud de \emph{text} y \( n \) es la longitud de \emph{pattern}. Retorna la posición donde inicia la primera ocurrencia de \emph{text} dentro de \emph{pattern}, o -1 si no existe.
\lstinputlisting[firstline=4, lastline=51]{./src/KMP.java}

\section{Programación dinámica}
	\subsection{Longest Increasing Subsequence}
		Halla la longitud de la subsecuencia creciente más larga que hay en un vector (o String). También halla los elementos que pertenecen a dicha subsecuencia, si fuese necesario. Su complejidad es \( O(n^2) \).
		\lstinputlisting[firstline=5, lastline=46]{./src/LIS.java}
		
	\subsection{Longest Common Subsequence}
		Halla la longitud de la subsecuencia común más larga entre dos Strings (o vectores). También halla los elementos que pertenecen a dicha subsecuencia, si fuese necesario. Su complejidad es \( O(mn) \), donde \( m \) y \( n \) son las longitudes de los Strings.
		\lstinputlisting[firstline=4, lastline=41]{./src/LCS.java}
	
	\subsection{Edit Distance}
	Halla el número mínimo de operaciones necesarias para transformar un String en otro. Las operaciones permitidas son eliminar, insertar o modificar un caracter del String. La complejidad del algoritmo es \( O(mn) \).
	\lstinputlisting[firstline=3, lastline=37]{./src/EditDistance.java}
	
	\subsection{Coin Change Problem}
	Se tienen monedas de \( n \) denominaciones diferentes. Se requiere encontrar el mínimo número de monedas tales que su valor sume exactamente \( W \). Su complejidad es \( O(nW) \).
	\lstinputlisting[firstline=5, lastline=59]{./src/CoinChange.java}	
	
	\subsection{El problema de la mochila (Knapsack)}
	Se tiene una mochila con capacidad \( W \), y \( n \) items con un peso \( w_i \) y un valor \( v_i \) cada uno. Se quiere hallar un conjunto de items tal que la suma de sus pesos no exceda \( W \), y que la suma de sus valores sea lo más grande posible. Su complejidad es \( O(nW) \). Es posible indicar cuál es el mayor valor posible, y con un ciclo adicional, indicar exactamente cuáles items se seleccionaron.
	\lstinputlisting[firstline=4, lastline=44]{./src/Knapsack.java}
	
\section{Range Minimum Query (Sparse Table)}
Dado un vector de \( n \) elementos, encuentra el mínimo elemento en el subarray comprendido entre los índices \emph{inf} y \emph{sup} (inclusivos).

Este algoritmo (Sparse Table) hace un pre-procesamiento del array en \( O(n \log n) \), y responde cada query en \( O(1) \).
	\lstinputlisting[firstline=3, lastline=48]{./src/SparseTable.java}

\section{Teoría de números}
	\subsection{Algoritmo de Euclides}
	Se utiliza para hallar el máximo común divisor (MCD) entre dos números. También se puede usar para hallar el mínimo común múltiplo (MCM).
	\lstinputlisting[firstline=3, lastline=15]{./src/Euclides.java}

	\subsection{Verificar si un número es primo}
	Dependiendo del problema, puede que nos sirva la forma ``fuerza bruta''. Esta forma tiene una complejidad de \( O(\sqrt{n}) \). Sin embargo, si tenemos números de más de 64 bits (que no caben en un \emph{long}) ya esta forma no es viable. 
	
	La clase BigInteger provee un método probabilístico para determinar si un número es primo. Si el número es compuesto, el método retona \emph{false} siempre. Si el método retorna \emph{true}, hay una probabilidad de \( 1-\frac{1}{2^x} \) de que el número sea primo, donde \(x\) es un parámetro que se le pasa a la función. Generalmente un valor de \(x = 10\) está bien.
	\lstinputlisting[firstline=5, lastline=25]{./src/CheckPrime.java}

	\subsection{Criba  de Eratóstenes}
	Algoritmo para hallar los números primos menores o iguales a \( n \). Su complejidad es \( O(n \log \log n) \).
	\lstinputlisting[firstline=7, lastline=28]{./src/Factorizar.java}
	
	\subsection{Factorización prima de un número}
	Se busca expresar un número \( n \) como una multiplicación de factores primos, de la forma:
	\[ n = \prod p_{i}^{a_{i}} = p_{1}^{a_{1}} \cdot p_{2}^{a_{2}} \cdot p_{3}^{a_{3}} ...  p_{k}^{a_{k}} \]
	
	Previamente se debe hacer una Criba de Eratóstenes modificada. Verifique en la especificación de la entrada del problema cuál es el máximo número \( x \) que tendrá que factorizar, y haga la Criba hasta \( \sqrt{x} \).
	
	El método retorna un HashMap donde la \emph{clave} es el factor primo \( p_{i} \) y el \emph{valor} su multiplicidad \( a_{i} \). Se puede modificar fácilmente para retornar una lista de todos los factores, o retornar la cantidad de factores.
	
	Funciona aproximadamente hasta \( n = 10^{14} \).
	\lstinputlisting[firstline=30, lastline=76]{./src/Factorizar.java}
	
	\subsection{Fórmulas}
	Para \( n \geq 2 \) es posible calcular algunas cosas partiendo de la factorización prima de \( n \):
	\[ n = \prod p_{i}^{a_{i}} = p_{1}^{a_{1}} \cdot p_{2}^{a_{2}} \cdot p_{3}^{a_{3}} ...  p_{k}^{a_{k}} \]
	\( n = 1 \) es un caso especial:
	\[ d(1) = \sigma (1) = \varphi (1) = 1 \]

		\subsubsection{Cantidad de divisores}
		\[ d(n) = \prod (a_{i} + 1) \]
		
		\subsubsection{Suma de divisores}
		\[ 
			\sigma (n) = \prod \frac{p_{i}^{a_{i} + 1} - 1}{p_{i} - 1}
		\]
		Esta función toma todos los divisores. Por ejemplo, los divisores de 12 son \( \{1, 2, 3, 4, 6, 12\} \). Por ende, \( \sigma(12) = 1+2+3+4+6+12 = 28 \). Si se quiere la suma de los divisores propios (es decir, los divisores excluyendo a \( n \)), basta con hallar:
		\[ 
			s (n) = \sigma (n) - n
		\]		
		En el ejemplo anterior, \( s (12) = 28-12 = 16 \).
		
		\subsubsection{Función $\varphi$ de Euler}
		Dos números son relativamente primos (o coprimos) si no tienen divisores en común (es decir, si su MCD es 1). \( \varphi (n) \) se define como la cantidad de enteros positivos menores a \( n \) y coprimos a \( n \). 
		\[ 
			\varphi (n) = \prod (p_{i} - 1) p_{i}^{a_{i}-1}
		\]

\section{Combinatoria}

	\subsection{Permutaciones}
	Un conjunto de \( n \) elementos diferentes tiene \( n! \) permutaciones. El número máximo cuyo factorial cabe en un \emph{long} de 64 bits es \( n = 20 \). Más allá, se requiere usar BigInteger.

\subsection{Subconjuntos}
	Un conjunto de \( n \) elementos tiene \( 2^{n} \) posibles subconjutos. 

	Si se busca generarlos, cada subconjunto puede representarse como un número \( b \) de \( n \) bits. El elemento \( k \) pertenece al subconjunto si el bit \( k \) de \( b \) está en 1. No es viable hacerlo para \( n > 30 \).
	\lstinputlisting[firstline=5, lastline=18]{./src/Subsets.java}

	\subsection{Coeficientes binomiales}
	El número de subconjuntos de tamaño \( k \) de un conjunto de tamaño \( n \), está dado por:
	\[ 
		\binom{n}{k} = \frac{n!}{k! \cdot (n-k)!} 
	\]

	Sin embargo no es muy práctico usar esta fórmula ya que involucra factoriales. Se puede utilizar la recurrencia \( \binom{n}{k} = \binom{n-1}{k-1} + \binom{n-1}{k} \) para generar todos los coeficientes binomiales \( \binom{i}{j} \) para \( 0 \leq i,j \leq n \).

	\lstinputlisting[firstline=4, lastline=14]{./src/CoefBinomial.java}

	Este código funciona para \( n \leq 67 \). Más allá, se requiere usar BigInteger.

	\subsection{Multiconjuntos}
	Un multiconjunto es un conjunto que permite elementos repetidos. El número de multiconjuntos de cardinalidad \( k \), con elementos tomados de un conjunto de cardinalidad \( n \), se puede contar como:
	\[ \multiset{n}{k} = \binom{k+n-1}{k} \]

	También se puede usar una recurrencia:
	\[ 
		\multiset{n}{0} = 1, \quad \multiset{0}{k} = 0 \; \textrm{para} \; k > 0 
	\]
	\[ 
		\multiset{n}{k} = \multiset{n}{k-1} + \multiset{n-1}{k} 
	\]

\subsection{Particiones}
	Una partición de un conjunto es una colección de subconjuntos disjuntos, tales que la unión de todos ellos es el conjunto original. La cantidad de maneras diferentes de particionar un conjunto de \( n \) elementos en \( k \) partes se denota como \( S(n, k) \) (números de Stirling de tipo 2).
	\[
		S(n,k) = \left \{ 
			\begin{array}{lcc}
			1 & n = 0
			\\ 1 & k = 0
			\\ S(n-1, k-1) + k \cdot S(n-1, k) & n \geq 1, k \geq 1
			\end{array}
		\right.
	\]

	\subsection{Desarreglos}
	Un desarreglo es una permutación que no ``mapea'' ningún elemento a sí mismo. Por ejemplo, 231 es un desarreglo de 123, pero 132 no lo es (ya que el 1 queda en la misma posición).

	\[
		D_{n} = \left \{ 
			\begin{array}{lcc}
				1 & n = 0
				\\ 0 & n = 1
				\\ (n-1)(D_{n-1} + D_{n-2}) & n \geq 2 
			\end{array}
		\right.
	\]
	
	El máximo \( D_n \) que cabe en un \emph{long} es con \( n=20 \).

	\subsection{Números de Catalan}
	\[
		C_{n} = \frac{1}{n+1} \binom{2n}{n}
	\]

	Sin embargo, puede ser más práctico usar la siguiente recurrencia:
	\[
		C_0 = 1, \quad C_{n} = \frac{2(2n-1)C_{n-1}}{(n+1)}
	\]

	El número \( C_{n} \) tiene muchas interpretaciones. Entre ellas:

	\begin{itemize}
		\item El número de árboles binarios diferentes con \( n \) vértices
		\item El número formas de hacer una expresión con \( n \) pares de paréntesis balanceados
		\item El número de formas de triangular un polígono con \( n+2 \) lados
		\item El número de caminos monótonos que no pasan sobre la diagonal en una cuadrícula de \( n \cdot n \).
	\end{itemize}

	El \( C_{n} \) más grande que cabe en un \emph{long} es con \( n = 33 \). 
		
\section{Otros}
	\subsection{Ordenamiento de Arrays y Listas}
	Cuando necesite ordenar un vector o una lista, utilice los métodos .sort() que tiene 		Java. El algoritmo que utilizan es QuickSort y su complejidad es \( O(n\log n) \).
	\lstinputlisting[firstline=7, lastline=15]{./src/Sorting.java}
	
	\subsection{Cola de Prioridad}
	Cola en la que la cabeza siempre es el menor elemento presente. Las operaciones .add() y .poll() son \( O(\log n ) \). La operación .peek() es \( O(1) \).
	
	También se puede invertir, haciendo que la cabeza siempre sea el mayor elemento, pasando el parámetro Collections.reverseOrder().
	\lstinputlisting[firstline=6, lastline=19]{./src/ColaDePrioridad.java}
	
	\subsection{Interfaz Comparable}
	En ocasiones se puede necesitar ordenar un vector o lista de un tipo de datos definido por el usuario (clase), o utilizar una cola de prioridad de este tipo. Para hacer esto, la clase debe implementar la interfaz Comparable de Java.

	El método compareTo() debe retornar:
	\begin{itemize}
		\item Negativo si \emph{this} \( < \) \emph{obj}.
		\item 0 si \emph{this} = \emph{obj}.
		\item Positivo si \emph{this} \( > \) \emph{obj}.
	\end{itemize}	
	
	Por ejemplo, se tiene una clase Persona con dos campos: nombre y edad. Se quiere ordenar una lista de objetos tipo Persona de menor a mayor edad.
	\lstinputlisting[firstline=4, lastline=23]{./src/Persona.java}
	
	\subsection{Búsqueda binaria}
	Java provee un método para hacer búsqueda binaria. Su complejidad es de \( O( \log n ) \), pero el vector (o lista) debe estar ordenado previamente.
	
	\lstinputlisting[firstline=7, lastline=14]{./src/BusquedaBinaria.java}
	
	Lo mismo pero con Collections.binarySearch() para buscar en ArrayList o LinkedList
	
	\subsection{Imprimir números decimales redondeados}
	Generalmente basta con esta función de Java para redondear correctamente números decimales.	
	\lstinputlisting[firstline=4, lastline=9]{./src/Redondear.java}
	
	\subsection{BufferedReader y BufferedWriter}
	\emph{Scanner} es sencillo de utilizar pero es lento. Se recomienda utilizar siempre \emph{BufferedReader} para leer entradas.
	
	En algunas ocasiones también se necesitará un modo más rápido que \emph{System.out.println()} para imprimr. \emph{BufferedWriter} es más rápido, nunca está de más usarlo.
	\lstinputlisting[firstline=9, lastline=29]{./src/ReaderWriter.java}

\end{multicols}	
\end{document}